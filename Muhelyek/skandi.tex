\documentclass{../styles/rulebook}

\begin{document}
\section*{ELTE Eötvös József Collegium \\ Skandinavisztika műhely\\ \vspace{0.5em} Műhelyszabályzat} 

\vspace{2em}

\section*{Preambulum}
TODO


\section{A műhely tagsága, tisztségei}

\begin{enumerate}
	\item Az Eötvös József Collegium (a továbbiakban Collegium) Skandinavisztika Műhelye (a továbbiakban Műhely) teljes jogú tagjának tekint minden fő- vagy mellékszakosként germanisztikai, azon belül skandinavisztikai tárgyú stúdiumokkal foglalkozó bentlakó és bejáró
	collegistát (abban az esetben is, ha az esetleges másik szakjához tartozó
	műhelyben aktív tevékenységet nem folytat), a műhelyvezetőt és annak
	helyettesét. 
	\begin{enumerate}
		\item A germanisztika egyéb területeivel foglalkozó (német és német
		nemzetiségi szakirányos) germanisták munkáját a Collegium Germanisztika
		Műhelye, illetve annak műhelyvezetője fogja össze és irányítja.
		\item A két műhely kiemelt együttműködését a Skandinavisztika Műhely azonban természetesen mindenkor kívánatosnak tartja, szorgalmazza és lehetőségei szerint támogatja.
	\end{enumerate}
	\item A műhelyvezető kötelességeit és jogait a Collegium Szervezeti és Működési
	Szabályzata (a továbbiakban SzMSz) tartalmazza. 
	\begin{enumerate}
		\item A Műhely vezetője helyettest nevezhet ki.
		\item A	műhelyvezető-helyettes feladata, hogy külső és belső fórumokon -- a
		műhelyvezetővel együtt, illetve annak akadályoztatása esetén a műhelyvezető
		meghatalmazásával -- képviselje a Műhelyt.
	\end{enumerate}
	\item A Műhely tagjainak kötelességeit és jogait az SzMSz, valamint a jelen Műhelyszabályzat és annak melléklete, a Curriculum tartalmazza.
	\item A Műhely collegista tagjai minden évben szótöbbséggel titkárt választanak
	maguk közül az őszi, lehetőleg szeptemberi műhelygyűlésen. 
	\begin{enumerate}
		\item A titkár a Műhely	hallgatóinak vezetője, részt vesz a szakmai felvételi vizsgán és nyilvános fórumokon képviseli a tagságot, illetve helyettesítheti a műhelyvezetőt.
	\end{enumerate}
	\item A Műhely collegista tagjai minden évben szótöbbséggel
	könyvtárost valasztanak maguk közül az őszi, lehetőleg szeptemberi műhelygyűlésen.
	\begin{enumerate}
		\item  A könyvtáros feladata a Műhely könyvtárának gondozása, az állomány
		nyilvántartása, ellenőrzése, az új tételek nyilvántartásba vétele, a műhelytagok
		diplomadolgozatainak őrzése,valamint a Műhely rendezvényeivel kapcsolatos anyagok
		archiválása.
	\end{enumerate}
	\item A Műhely collegista tagjai műhelyhonlap-felelőst nevezhetnek
	ki. 
	\begin{enumerate}
		\item A honlapfelelős feladata a Műhely honlapjának gondozása, fejlesztése és
		legalább havi szinten történő folyamatos frissítése.
	\end{enumerate}
	\item A Műhely minden szemeszterben legalább két alkalommal, illetve optimális esetben
	havi rendszerességgel műhelygyűlést tart, melyeken a Műhely tagjai kötelesek
	megjelenni. 
\end{enumerate}


\section{Szakmai munka}

\begin{enumerate}
	\item A Műhely tanegységeinek rendszerét a műhelyvezető a tagok véleményét meghallgatva alakítja ki. 
	\item A műhelytagoknak minden félévben kötelező legalább egy, a Műhely által meghirdetett kurzust elvégezni, valamint a választott idegen nyelvhez/nyelvekhez kapcsolódó collegiumi
	nyelvórát/nyelvórákat felvenni és látogatni.
	\item A szakmai kurzusokkal kapcsolatos
	részletes követelményeket a Műhely Curriculuma tartalmazza.
	\item A Műhely tagjai legkésőbb tanulmányaik befejezésekor
	kötelesek (legalább) két B2-es szintű komplex nyelvvizsgával
	rendelkezni 
	\begin{enumerate}
		\item Ebbe a kitételbe a korábban megszerzett nyelvvizsgák beszámítandók. 
		\item Csak skandinavisztika szakosok esetében a második idegennyelvi nyelvvizsgára a 8.
		félév végéig kerül sor, számukra egy további nyelv középfokú ismerete ajánlott.
		\item A két nyelvi szakon tanulmányokat folytató hallgatók esetében egy további
		idegen nyelv középfokú ismerete javallott, azonban harmadik nyelvből nyelvvizsgát
		szerezniük azonban nem kötelező. 
		\item A tanult nyelvek szabadon választhatók, szakmai és egyéb szempontokból főleg az angol és a francia, illetve valamely más germán nyelv legalább középfokú ismerete indokolt.
	\end{enumerate}
	\item A Műhely által meghirdetett tanegységeket a műhelyvezető, illetve az adott kurzus oktatója jóváhagyásával minden collegista felveheti.
	\begin{enumerate}
		\item Túljelentkezés esetén a Műhely tagjai elsőbbséget élveznek. 
		\item A szemináriumokon a mindenkori tanár engedélyével (korlátozott számban) külsős	hallgatók is részt vehetnek. 
		\item A collegiumi órák helyszíne (különleges alkalmakat leszámítva) a Collegium épülete.
	\end{enumerate}	
	\item Ajánlott, hogy a Műhely collegista tagjai szakmai munkájukat (szakdolgozat, pályázati részvétel, témavezetés) egyeztessék a
	műhelyvezetővel.
	\item A skandinavisztikai témában szakdolgozó műhelytagok
	szakdolgozatukat kötelesek a Műhely előtt tartott előadásban bemutatni és
	műhelyvitára bocsátani. 
	\begin{enumerate}
		\item A megvédett szakdolgozat egy kötött példánya a Műhely könyvtárába kerül.
	\end{enumerate}
	\item A Műhely tagjai számára javallott öt éves collegiumi tagságuk
	ideje alatt 
	\begin{enumerate}
		\item legalább egyszer OTDK-dolgozatot készíteni 
		\item legalább kétszer előadást tartani az évente megrendezett Eötvös-konferencián vagy más egyéb szakmai fórumon
		\item legalább két önálló tanulmányt publikálni.
	\end{enumerate}
	\item A Műhely tagjainak collegiumi tagságuk ideje alatt ajánlott legalább egy három hetes -- optimális esetben féléves --, célnyelvi nyelvterületre szóló külföldi ösztöndíj (nyári egyetem, Erasmus, népfőiskola stb.) megszerzése.
	\item A Műhely doktorandusz-senior státusú tagjai kötelezhetők félévente egy, a szakterületükkel (irodalomtudomány, nyelvtudomány, kultúratudomány stb.) kapcsolatos szakmai szakszeminárium, illetve igény szerint alapozó kurzus (proszeminárium) megtartására.
	\item A Műhely collegista tagjai számára szakmai előmenetelük
	érdekében kívánatos, hogy -- lehetőség szerint aktívan, de legalább hallgatóként -- részt vegyenek a Műhely által szervezett szakmai rendezvényeken, ápolják és bővítsék a Műhely külkapcsolatait, erejükhöz és tehetségükhöz mérten erősítsék annak szakmai pozícióit, illetve minden körülmények között méltón képviseljék az Eötvös Collegium szellemiségét.
\end{enumerate}


\section{Zárórendelkezések}

\begin{enumerate}
	\item Jelen Műhelyszabályzat a csatolt Curriculummal együtt érvényes.
	\item A Műhely jelen szabályzata a 2014/2015. tanév II. félévétől lép érvénybe.
\end{enumerate}
\end{document}