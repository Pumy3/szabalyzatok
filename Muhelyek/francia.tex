\documentclass{../styles/rulebook}

\begin{document}
\section*{ELTE Eötvös József Collegium \\ Aurélien Sauvageot Francia műhely\\ \vspace{0.5em} Műhelyszabályzat} 

\vspace{2em}

\section*{Preambulum}
TODO


\section{A műhely tagsága, tisztségei}

\begin{enumerate}
	\item Az Aurélien Sauvageot Francia Műhely (a továbbiakban Műhely) tagja minden régi (kredites) rendszerű képzésben részesülő francia szakos romanisztika (francia szakirányú) alapszakos, francia nyelv, irodalom és kultúra mesterszakos, illetve a Fordító és Tolmácsképző Intézetben francia mesterszakos bentlakó és bejáró collegista, valamint a műhelyvezető és a vezető tanár.
	\item A műhelyvezető kötelességeit és jogait az Eötvös József Collegium (a továbbiakban Collegium) Szervezeti és Működési Szabályzata (a továbbiakban SzMSz) tartalmazza.
	\item A Műhely vezetőjének munkáját a vezető tanár segíti, külső és belső fórumokon műhelyvezetővel együtt képviseli a műhelyt, illetve kapcsolatot tart a DEA-és az ERASMUS-ösztöndíjakat odaítélő magyar es francia egyetemi intézményekkel.
	\item A Műhely tagjainak kötelességeit és jogait az SzMSz, valamint jelen Műhelyszabályzat tartalmazza.
	\item A Műhely collegista tagjai minden évben szótöbbséggel az őszi műhelygyűlésen titkárt választanak maguk közül. A titkár a Műhely hallgatóinak vezetője, részt vesz a szakmai felvételi vizsgán, s a nyilvános fórumokon képviseli a műhelytagokat.
	\item A Műhely minden évben legalább két (szemeszterenként egy) műhelygyűlést tart, melyeken a Műhely tagjai kötelesek megjelenni.
\end{enumerate}


\section{Szakmai munka}

\begin{enumerate}
	\item A Műhely legfontosabb célja az egyetemi francia irodalmi, nyelvészeti és civilizáció-oktatás BA, MA, minor stb. képzési formáitól független kiegészítése, illetve egy-egy speciális területének elmélyítése. Ebből következően a collegistáknak többet kell teljesíteniük nem collegista évfolyamtársaikhoz képest. A többletteljesítés kereteit jelen Szabályzat határozza meg.
	\item A Műhely BA-képzésben részesülő collegista tagjai kötelesek (legkésőbb) az év végén minor szakot választani.
	\item A Műhely szervesen illeszkedik a Collegium minőségi szakképzési struktúrájába, amely elsősorban a kiscsoportos és a tutoriális intenzív oktatás pedagógiai módszereire épül.
	\item A Műhely tanegységeit a műhely vezetője a tagok véleményét meghallgatva alakítja ki. 
	\item A collegisták a (lehetőség szerint) minden félévben meghirdetett főszemináriumot, a képzési szintjüknek megfelelő szakszemináriumo(ka)t, valamint a választott idegen nyelvhez kapcsolódó nyelvórá(ka)t kötelesek felvenni és látogatni.
	\item Az (5) pontban felsorolt szakszemináriumokon kívül a meghívott hazai és külföldi vendégtanárok tematikus előadássorozatai lehetővé teszik a műhelymunka irodalomelméleti és -történeti, filozófiai, társadalomtörténeti, illetve nyelvészeti elmélyítését.
	\item A Műhely által meghirdetett tanegységeket minden collegista felveheti.
	\begin{enumerate}
		\item A szakszemináriumokon a tanár engedélyével (korlátozott számban) külső hallgatók is részt vehetnek. \item A collegiumi órák helyszíne (különleges alkalmakat leszámítva) a Collegium épülete.
	\end{enumerate}
	\item A Műhely tagjai -- az SzMSz rendelkezéseivel összhangban -- tanulmányaik ideje alatt kötelesek (legalább) két középfokú C-típusú nyelvvizsgát tenni. 
	\begin{enumerate}
		\item Az első nyelvvizsgára a 4., a második nyelvvizsgára a 10. félév végén kerül sor.
	\end{enumerate}
	\item A Műhely collegista tagjai szakmai munkájukat (szakdolgozat, pályázati részvétel, témavezetés) kötelesek egyeztetni a műhelyvezetővel. Kívánatos, hogy a hallgatót bármilyen formában segítő oktató a Collegiumhoz kötődő személy, volt collegista legyen.
	\item A Műhely tagjai számára szakmai előmenetelük érdekében kívánatos, hogy részt vegyenek a Collegiumban rendezett, az egyes francia tudományterületekhez kötődő tematikus konferenciákon, (mű)fordítói szemináriumokon és szakmai eszmecseréken.
\end{enumerate}


\section{Zárórendelkezések}

\begin{enumerate}
	\item Jelen szabályzat a 2009/2010-es tanév tavaszi félévének első napjától hatályos.
	\item Jelen szabályzat a Collegium Oktatási, Tanulmányi Szabályzatának és Követelményrendszerének mellékletét képezi.
\end{enumerate}

\end{document}