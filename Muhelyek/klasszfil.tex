\documentclass{../styles/rulebook}

\begin{document}
\section*{ELTE Eötvös József Collegium \\ Klasszika-filológia Officina\\ \vspace{0.5em} Műhelyszabályzat} 

\vspace{2em}

\section*{Preambulum}

Az ELTE Eötvös József Collegium (a továbbiakban Collegium) Bollók János Klasszika-filológia Officinája (a továbbiakban Officina) Szabályzatának bevezetőjében négy gondolatot kívánunk megfogalmazni:

\begin{enumerate}[label=\Roman*.]
\item Kedves tanárunk és tanártársunk, a Collegium egykori igazgatója, Bollók János iránt érzett tiszteletünk jeléül – aki a Collegium újkori történetében hallgatók nemzedékeit tanította, és e Műhelyt megújította – a Klasszika-filológia Műhelyt Bollók János Klasszika-filológia Officinának nevezzük.
\item A klasszika-filológiai képzés a Collegium klasszikus- (1895–1950) és újkori történetében egyaránt a Collegiumban folyó hagyományos (tudós) tanárképzés meghatározó része volt. Célunk e hagyomány folytonosságának fenntartása, ezért lehetőségeinkhez mérten e történelmi távlatú szakképzés szellemiségét követjük (vö.: Borzsák István: A klasszika-filológia az Eötvös Collegiumban).
\item Az Officina legfontosabb célja az egyetemi klasszika-filológia, ógörög, latin és újgörög oktatás képzési formától (BA, MA, minor stb.) független kiegészítése (beleértve a latin, illetve ógörög köteles szakok követelményeit is), melyből egyértelműen következik, hogy a collegistának többet kell teljesítenie nem collegista évfolyamtársához képest. A többletteljesítmény kereteit jelen szabályzat határozza meg.
\item A Bollók János Klasszika-filológiai Officina -- a Collegium Szervezeti és Működési Szabályzatának (a továbbiakban SzMSz) rendelkezéseivel összhangban és külföldi kapcsolataira való tekintettel -- kiemelten fontosnak tekinti, hogy tagjai a collegiumi nyelvoktatás lehetőségeit kihasználva tanulmányaik 10 félévének végére két középfokú C-típusú nyelvvizsgát szerezzenek.

\end{enumerate}


\section{A műhely tagsága, tisztségei}

\begin{enumerate}
	\item Az Officina tagja minden klasszika-filológia, latin, ógörög, újgörög szakos bentlakó és bejáró collegista, a tiszteletbeli elnök, valamint a műhelyvezető és helyettese.
	\item Az Officina a Collegiumhoz kötődő legtekintélyesebb magyar klasszika-filológusok közül tiszteletbeli elnököt választhat. A tiszteletbeli elnök az Officina valamennyi fórumán (órák, gyűlések, vizsgák stb.) teljes jogú tagként vehet részt.
	\item A műhelyvezető kötelességeit és jogait az SzMSz tartalmazza.
	\item Az Officina műhelyvezetője helyettest nevezhet ki. A műhelyvezető-helyettes feladata, hogy külső és belső fórumokon képviselje az Officinát a műhelyvezetővel együtt, illetve annak akadályoztatása esetén.
	\item Az Officina hallgatóinak kötelességeit és jogait az SzMSz, valamint jelen szabályzat tartalmazza. 
	\item Az Officina collegista tagjai minden évben szótöbbséggel titkárt választanak maguk közül az őszi műhelygyűlésen. A titkár az Officina hallgatóinak vezetője, részt vesz a szakmai felvételi vizsgán, nyilvános fórumokon képviseli a tagságot.
	\item Az Officina collegista tagjai minden évben szótöbbséggel könyvtárost választanak maguk közül az őszi műhelygyűlésen. A könyvtáros feladata az Officina könyvtárának gondozása, nyilvántartása, ellenőrzése, a hallgatók szemináriumi dolgozatainak őrzése, az Officina rendezvényeivel kapcsolatos anyag archiválása.
	\item Az Officina minden évben (legalább) két (szemeszterenként egy) Műhelygyűlést tart, melyeken az Officina tagjai kötelesek megjelenni.
\end{enumerate}


\section{Szakmai munka}

\begin{enumerate}
	\item Az Officina BA-képzésben tanuló collegista tagjai kötelesek (legkésőbb) az első év végén minor szakot választani.
	\item Az Officina collegista tagjai tutori rendszerben végzik tanulmányaikat. Az Officina tagsága az őszi műhelygyűlésen valamennyi felvett hallgató mellé tutort választ. Kívánatos, hogy a tutor felsőbb éves tanulmányokat folytató hallgató legyen. A tutor feladata, hogy a tutorált szakmai munkáját segítse, ellenőrizze, beilleszkedését támogassa.
	\item Az Officina tanegységeit az Officina vezetője a tagok véleményét meghallgatva alakítja ki (lásd: képzési rend - melléklet). A collegisták a (lehetőség szerint) minden félévben meghirdetett főszemináriumot, a képzési szintjüknek megfelelő szakszemináriumo(ka)t, valamint a választott idegen nyelvhez kapcsolódó nyelvórá(ka)t kötelesek felvenni illetve látogatni.
	\item Az Officina tagjai (az SzMSz rendelkezéseivel összhangban) tanulmányaik ideje alatt kötelesek (legalább) két középfokú C-típusú nyelvvizsgát szerezni. Az első nyelvvizsgára a 4., a második nyelvvizsgára a 10. félév végén kerül sor. Az Officina külföldi kapcsolataira való tekintettel ajánlott nyelv a német, az angol és a francia.
	\item Az Officina által meghirdetett tanegységeket minden collegista felveheti. A szemináriumokon a tanár engedélyével (korlátozott számban) külsős hallgatók is részt vehetnek. A collegiumi órák helyszíne (különleges alkalmakat leszámítva) a Collegium épülete.
	\item Az Officina collegista tagjai szakmai munkájukat minden vonatkozásban (szakdolgozat, pályázati részvétel, témavezetés) kötelesek egyeztetni a műhelyvezetővel. Kívánatos, hogy a hallgatót bármilyen formában segítő oktató a Collegiumhoz kötődő személy, volt collegista legyen.
	\item Az Officina collegista tagjai automatikusan tagjai a Collegium Hungaricum Societatis Europaeae Studiosorum Classicae Philologiae (a továbbiakban SEC) szakmai diákszervezet \emph{Eötvös József Collegium} tagszervezetének. Tanulmányaik első két évében hallgatóként, a harmadik évtől előadóként kötelesek részt venni a tagszervezet felolvasó ülésein, a SEC országos rendezvényein.
	\item Az Officina collegista tagjai számára szakmai előmenetelük érdekében kívánatos, hogy belépjenek tudományszakunk országos egyesületeibe, a Magyar Ókortudományi Társaságba, illetve a Magyar Bizantinológiai Társaságba, azok rendezvényein lehetőség szerint részt vegyenek, képviseljék a Collegium szellemiségét.
\end{enumerate}


\section{Zárórendelkezések}

\begin{enumerate}
	\item Jelen szabályzat a 2009/2010-es tanév őszi félévétől lép életbe.
	\item Jelen szabályzat az ELTE Eötvös József Collegium Oktatási, Tanulmányi Szabályzat és Követelményrendszer mellékletét képezi.
\end{enumerate}

\end{document}